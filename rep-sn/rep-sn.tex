\documentclass[12pt]{article}
\usepackage{blubird}
% has options [nodate, nosans, nofancy, nocolor, code]

% TITLE
\title{How to Represent a Permutation \\ \small{and why i drew boxes all summer}}
\author{Bryan Lu}
\date{7 November 2023} % do not use if [nodate] option enabled

\begin{document}
\maketitle

\begin{abstract}
  Representation theory describes a broad strategy of analyzing
  algebraic structures by understanding how they act on vector spaces. This allows
  us to employ the tools of linear algebra to make studying abstract algebraic
  objects more tractible. We will
  look at some basic tools in representation theory as applied to the symmetric
  group, and talk about connections to combinatorics. (Time-permitting, we may also
  look at applications of the theory to current research in
  algebraic combinatorics!)
\end{abstract}

pretext: 
\begin{itemize}
  \item ask: how much familiarity with linear algebra? today all of our vector
    spaces will be \emph{complex} (algebraically closed, characteristic zero)  
  \item ask: group theory? 
\end{itemize}

\section{Context}
\begin{itemize}
  \item what is a representation? why study representation theory?
  \item philosophy: studying the structure of objects by studying how they act
    on other things
  \item touches algebra (naturally), but also geometry and combinatorics
  \item acting on vector spaces is nice! in particular, we know how to study
    linear transformations on vector spaces (just matrices and linear algebra)
\end{itemize}


\section{Basics}
\begin{definition*}
  A (finite-dimensional) \emph{representation} of a group $G$ is: 
\begin{itemize}
  \item a group homomorphism $\rho : G \to \GL_n(\CC)$.  
  \item a $\CC[G]$-module $M$. (abelian group with action of $\CC[G]$, much like
    a generalization of a vector space.)

    $\CC[G]$
    can be thought of the set of formal complex linear combinations of $G$,
    and you can think of this as a function $f : G \to \CC$, where a
    linear combination corresponds to ``$\sum_{g \in G} f(g) g$''. 
\end{itemize}
\end{definition*}

\begin{example*}
  Consider the dihedral group $D_3$ acting on the plane via rotations and
  reflections (can view it as $\RR^2$ inside $\CC^2$, really):
  \[
    D_3 = \tuple{r, s \mid r^3 = s^2 = 1}
  \]
  \[
    r = \bmat{-\frac{1}{2} & -\frac{\sqrt 3}{2} \\ \frac{\sqrt 3}{2} &
    -\frac{1}{2}} \qquad s = \bmat{1 & 0 \\ 0 & -1}
  \] 
\end{example*}

\begin{example*}
  Look at all one-dimensional representations of $D_3$. 

  The trivial representation of $D_3$ -- everything acts as identity! 
  Also, thinking of $D_3 = S_3$, the sign representation 

  Also, the standard representation in matrices (embed this as acting on a space with
  3 basis vectors)
\end{example*}

\begin{example*}
  The regular representation of any finite group $G$ -- one basis element for
  every element in the group. 
\end{example*}

\begin{definition*}
  A \emph{subrepresentation} of a representation:
  \begin{itemize}
    \item $\rho : G \to \GL_n(\CC)$ is a subspace $W \subseteq \CC^n$ preserved
      by $\rho(g)$ for all $g \in G$.
    \item $M$ is a submodule of $M$. 
\end{itemize}  
\end{definition*}

\begin{definition*}
  A representation is \emph{irreducible} if there are no proper
  subrepresentations (i.e. if $\rho$ is a representation of $G$, then $\rho(g)$
  has no nontrivial invariant subspaces across all $g$, and if $M$ is a representation of $G$, it is a simple
  $\CC[G]$-module.) 
\end{definition*}

Very important theorems:
\begin{theorem*}[Schur's Lemma]
  Any intertwining map of irreducible representations is either the zero map (if
  these representations are not isomorphic) or a scalar multiple of the identity.
  $f : V \to W$, $\rho_V$, $\rho_W$ reps, for any $g \in G$, $f \circ \rho_V(g)
  = \rho_W(g) \circ f$. 

  \noindent\textbf{NOT true if the field is not algebraically closed!}
\end{theorem*}

\begin{theorem*}[Maschke's Theorem (+ Schur's Lemma)]
  Any representation of $G$ over $\CC$ has a (unique) decomposition into a
  direct sum of irreducibles. 

  \noindent\textbf{NOT true in fields with positive characteristic (or
  algebraically closed!)}
\end{theorem*}

\section{Characters}
\begin{definition*}
  For $G$ a group and $\rho : G \to \GL(V)$ a representation of a group $G$ on
  $V$, the \emph{character} of $\rho$ is the function $\chi_\rho : G \to \CC$
  given by $\chi(g) = \tr(\rho(g))$.  
\end{definition*} 
The theory of characters is very rich! skipping over a lot of proofs.

Let $\rho$ be a representation of $G$ over $\CC^n$, and $\chi$ the associated
character. Some facts about characters: 
\begin{itemize}
  \item $\chi(1) = n$, the dimension of the ambient space
  \item $\chi(g^{-1}) = \ol{\chi(g)}$
  \item for another representation $\sigma$, $\chi_\rho + \chi_\sigma =
    \chi_{\rho \oplus \sigma}$, and $\chi_\rho \cdot \chi_{\sigma} =
    \chi_{\rho\otimes \sigma}$. 
  \item $\chi$ is constant on conjugacy classes
\end{itemize}

\begin{definition*}
  The \emph{character table} of a group $G$ is a square table with the rows
  being irreducible representations of $G$ and the columns the conjugacy classes
  of $G$, and each entry the value of that irrep on that conjugacy class.  
\end{definition*}

\begin{example}
Symmetric group $D_3 = S_3$ and $S_4$:  
\begin{center}
\begin{tabular}{c|ccc}
    $S_3$ & (1) & (12) & (123) \\ \hline 
    trivial, (3) & 1 & 1 & 1 \\ 
    sign, (1,1,1) & 1 & -1 & 1 \\ 
    standard, (2,1) & 2 & 0 & -1 \\ 
  \end{tabular}
\end{center}  
\end{example}


\begin{definition*}
  A \emph{class function} on $G$ is a complex-valued function on $G$ which is
  constant on conjugacy classes. Note that characters are class functions, and
  inside the group ring (thought of as functions), these commute with all
  elements under multiplication.  
\end{definition*}

Following from Schur's Lemma, irreducible characters are very very special: 
\begin{theorem*}
  The set of class functions on $G$ forms a complex vector space, on which we
  can define the following Hermitian inner product:
  \[
    \ip \vphi \psi = \frac{1}{|G|} \sum_{g \in G} \ol{\vphi(g)} \psi(g). 
  \] 
  Irreducible characters form an orthonormal basis for this vector space. 
  In particular, if $\rho_i, \rho_j$ are irreps of $G$, and $\chi_i$ and $\chi_j$
  are the associated irreducible characters, then 
  \[
    \ip {\chi_i}{\chi_j} = \begin{cases}
      1 & \rho_i \iso \rho_j \\ 0 & \rho_i \not\iso \rho_j
    \end{cases}
  \]
  (check this by example)
\end{theorem*}

\begin{corollary*}
  $\rho$ is irreducible iff $\ip \chi \chi = 1$. 
\end{corollary*}

\begin{corollary*}
  The number of irreps of $G$ is equal to the number of conjugacy classes of
  $G$. 
\end{corollary*}

\begin{corollary*}
  Isomorphic representations have the same character.
\end{corollary*}

\begin{corollary*}
  The columns of the character table are orthogonal under the standard inner
  product. Precisely, if $s$ and $t$ are in different conjugacy classes, then
  $\sum_{i=1}^r \chi_i(s) \ol{\chi_i(t)} = 0$, where this sum goes over all
  irreps.  
\end{corollary*}

\begin{example*}
  Character of the regular representation can be computed -- $\chi_G(1) = |G|$,
  and $\chi_G(g) = 0$ for $g \neq 1 \in G$.

  Projecting this in terms of the irreducibles $\chi_i$, we have that 
  \[
    \ip {\chi_i}{\chi_G} = \frac{1}{|G|} \ol{\chi_G(1)} \chi_i(1) = n_i
  \] 
  where $n_i$ is the dimension of the irrep. Therefore, $\chi_G = \sum_{i=1}^r
  n_i \chi_i$ as we sum over all irreps, and by dimension counting we have that
  $|G| = \sum_{i=1}^r n_i^2$. 
\end{example*}

Let's use this information to figure out all of the irreps of $S_4$ by their
characters.
\begin{itemize}
  \item Take trivial and sign representation 
  \item look at the standard representation, it's not irreducible but getting
    rid of $e_1 + e_2 + e_3 + e_4$, it then becomes so! (trivial rep)
  \item take the tensor product of this with the sign rep
  \item Work out the last irrep from dimension counting and then orthogonality 
    (by columns is easiest) 
\end{itemize}

\begin{center}
  \begin{tabular}{r|ccccc}
    $S_4$ & (1) & (12) & (12)(34) & (123) & (1234) \\ \hline
    trivial (4) & 1 & 1 & 1 & 1 & 1 \\ 
    sign (1, 1, 1, 1) & 1 & -1 & 1 & 1 & -1 \\ 
    standard (3, 1) & 3 & 1 & -1 & 0 & -1 \\ 
    dual standard (2, 1, 1) & 3 & -1 & -1 & 0 & 1 \\ 
    last (2, 2) & 2 & 0 & 2 & -1 & 0
  \end{tabular}
\end{center}

\section{Specht Modules} 
Note that a permutation $\sigma \in S_n$ can be decomposed into a product of
cycles, and the conjugacy classes of $S_n$ are indexed by partitions of $n$
(into cycles). Therefore, the irreducible representations are also in bijection
with the partitions of $n$ -- and in particular, we can construct them for every
partition of $n$ explicitly!

Fix your favorite partition $\lambda \vdash n$. Draw as a Young diagram.  
\begin{itemize}
  \item Look at every \emph{standard Young tableau} of shape $\lambda$. Put the
    numbers 1-$n$ in boxes of $\lambda$, increasing down and across. 
  \item Consider the \emph{tabloid} for every SYT $t$ of shape $\lambda$ (giving
    a tabloid $\set t$, where we take equivalence classes of rows. 
  \item For every permutation $\sigma$ of the numbers that preserves the columns
    of $t$, apply $\sigma$ to $\set t$ and keep a running sum of
    $\sgn(\sigma)\set{\sigma \cdot t}$. This gives the \emph{associated
    polytabloid} for $t$.  
\end{itemize}
The complex vector space spanned by these associated polytabloids is called the
\emph{Specht module} corresponding to $\lambda$. 

\begin{example*}
  $\lambda = (2, 2)$.
\end{example*}


\section{Symmetric Functions}
\begin{definition*}
  A \emph{symmetric function} $f$ in $n$ variables $x_1, \dots, x_n$ is a
  polynomial in those variables such that for any permutation $\sigma \in S_n$,
  $f(x_1, \dots, x_n) = f(x_{\sigma(1)}, \dots, x_{\sigma(n)})$. 
\end{definition*}
Several families: 
\[
  p_k = \sum_{i=1}^n x_i^k
\] 
\[
  h_k = \sum_{l_1 + \dots l_n = k, l_i \geq 0} x_1^{l_1} \dots x_n^{l_n}
\]
Special: \emph{Schur polynomials}, a positive basis for products of symmetric
functions.  
\[
  s_\lambda = \sum_{T \in SSYT(\lambda)} x_T, \qquad x_T = \prod_{i \in T} x_i
\] 

\begin{definition*}
  For any class function on $S_n$, $f$, define $\mathrm{ch}(f) = \frac{1}{n!}
  \sum_{\sigma \in S_n} f(\sigma) p_{c(\sigma)}$ , where $c(\sigma)$ is the
  partition that $\sigma$ defines (its cycle type) and $p_\lambda =
  p_{\lambda_1} \dots p_{\lambda_k}$ 
\end{definition*}

\begin{theorem*}
  The map $\mathrm{ch}$ from the ring of class functions of $S_n$ to the
  symmetric functions is an isomorphism.  
\end{theorem*}
Manipulating representations of $S_n$ is as easy as manipulating polynomials!

\section{Why?}
In algebraic combinatorics, it's not uncommon to study rings associated to
certain combinatorial structures, which might admit a natural group action. (For
instance, the Stanley-Reisner ring associated to a simplicial complex). One
example that I care a lot about, a quotient of the Stanley-Reisner ring of the
order complex of the Boolean lattice: 
\[
  k[\Delta B_n] = \frac{k[x_F : F \in B_n]}{\tuple{x_P x_Q : P, Q \text{
  incomparable}}}
\] 
\[
  C(B_n) := \frac{k[\Delta B_n]}{\tuple{\Theta_{colorful}}}, \Theta_{colorful} =
  \set{ \sum_{|F| = i} x_F, 0 \leq k \leq n}
\] 
When $k = \CC$ (we usually want it to be a field) each graded component is a
vector space, admitting an $S_n$-action $\sigma \cdot x_F = x_{\sigma(F)}$. This
then makes each graded component an $S_n$-representation!

Different graded components of this family of rings are related to each other by
restricting representations of $S_n$ to the subgroup $S_{n-1}$. In the course of
figuring this out we see that: 
\begin{theorem}[DHKLT 2023]
For $\lambda / \mu$ a ribbon tableau with $n$ boxes, 
\[
  s_{\lambda / \mu}\downarrow^{S_n}_{S_{n-1}} = \text{you know what it is} 
\] 
\end{theorem}






\end{document} 
