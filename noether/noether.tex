\documentclass[12pt]{article}
\usepackage{blubird}
% has options [nodate, nosans, nofancy, nocolor, code]

% TITLE
\title{Noether's Theorem}
\author{Bryan Lu}
\date{9 June 2023} % do not use if [nodate] option enabled

\begin{document}
\maketitle

\section{Introduction}
In 1915, mathematician Emmy Noether proved a theorem in mathematical
physics that has provided critical insights into developing new theory and 
analyzing physical systems in the 20th century. It can be concisely stated as
follows: 
\begin{center}
  \tit{For every \tbf{continuous symmetry} of a system, there is a corresponding
  \tbf{conserved quantity}.} 
\end{center}
This is kind of like vague and confusing though -- what does this mean, and how
do we get it? 

\section{Lagrangian Mechanics}
To understand this statement, we have to transition to a different formalism for
mechanics than the typical formalism presented in physics. In a nutshell,
instead of the ``forces cause motion'' perspective on mechanics, we instead
consider a general principle guiding a system, i.e. the idea that ``nature 
takes the path of `least resistance.''' This is sort of imprecise,
but it's a useful line of reasoning -- we consider the \tbf{action} $S =
\int_{t_1}^{t_2} \lagr \, dt$ for a quantity $\lagr$ called the \tbf{Lagrangian} that is
expressed in the coordinates (and their time derivatives) of the system, i.e.
$\lagr
= \lagr(q_1, \dots, q_n, \dot q_1, \dots, \dot q_n, t)$, and we postulate instead that
the action must be minimized (\tbf{Hamilton's principle}). This mathematical
idea leads to the \tbf{Euler-Lagrange equations},
which tells us if the $q_i$s (as functions of time, $t$) describe the actual
motion of the system, then $\dv{}t \left( \pdv \lagr {\dot q_i} \right) - \pdv
\lagr{q_i} = 0$. It turns out that when $\lagr = T - V$, where $T$ is the kinetic
energy of the system and $V$ is the potential energy of the system, we can
recover the normal equations of motion a la Newtonian mechanics. For example, in
one variable, applying Euler-Lagrange: 
\[
  \lagr = T - V = \frac{1}{2} m \dot x ^2 - V(x) \Rightarrow \dv{}{t} \left(
    m \dot x \right) + \dv V x = 0 \Rightarrow m \ddot x = - \dv V x 
\] 
we can easily recover Newton's second law! In particular, reasoning from
Newton's laws, we can show that if we have conservative forces, this form of
$\lagr$ is the correct one. 

In general, the coordinates $q_i$ can represent coordinates that are not
necessarily ``dimensions of length,'' i.e. charge, angle, etc. also work. For
instance, if we do things in the plane in polar coordinates, then: 
\[
  \lagr = T - V = \frac{1}{2} m (\dot r^2 + r^2 \dot \theta^2) - V(r, \theta) 
\] 
\[
  \dv{} t (m \dot r) - mr \dot \theta^2 = - \pdv V r \qquad \dv{}t (mr^2 \dot
  \theta) = - \pdv V \theta 
\] 
and now we proceed as before, solving these differential equations. Note the
appearance of the acceleration in polar coordinates appearing in the correct
forms on the left side, and the force on the right. 

We can see then that the Lagrangian is a way for us to \tit{describe our system}
compactly in a single function, and its forms may vary based on accounting for
other forces (i.e. electromagnetism).

\section{Noether's Theorem}
One of the most important tools that we were able to use when solving problems before was with conserved quantities, such as linear momentum, angular momentum, and energy - but it's not really clear how they arise in Lagrangian mechanics. 

Since the Lagrangian we know describes our system, we now describe what we mean
by a continuous symmetry. Remember that the coordinates $q$ are their own
functions of time $t$. Suppose that we can model a transformation of $q$ as a
function $\tilde q(s, t)$, so that the coordinates can also be changed
as a continuous function of a different parameter, $s$, and when $s = 0$,
$\tilde q(0, t) = q(t)$ is left invariant. When we replace all instances of $q$ with
$\tilde q$ in our Lagrangian, let's see how to vary $s$ very
slightly in a way that leaves the Lagrangian invariant. If we can, then this actually induces a conserved quantity, which will
follow from some mathematical computation. For any symmetry, we 
specifically want for our definition that the
Lagrangian is constant up to first order, so we have $\dv{\lagr}{s}\Big|_{s=0} =
0$. This means we only need to consider first-order changes in $q$ as well, so
assume $\tilde q(s, t) \approx q(t) + s\xi(t)$:  
\[
  0 = \dv{\lagr}{s}\Big|_{s=0} 
  = \left(\pdv{\lagr}{\tilde q} \dv{\tilde q}{s} + \pdv{\lagr}{\dot{\tilde q}}
    \dv{\dot{\tilde q}}{s}\right)\Big|_{s=0}  
    = \dv{}{t} \left(\pdv{\lagr}{\dot{q}} \right) \xi +
  \pdv{\lagr}{\dot{q}} \dv{\xi}t   \\ 
    = \dv{}{t} \left(\pdv{\lagr}{\dot{q}} \xi \right) 
\]
so it seems that the quantity $\pdv \lagr {\dot q} \xi$ is constant!
This quantity is sometimes called the ``conserved charge'' from field theory,
but I'd particularly like to draw your attention to $\pdv \lagr {\dot q}$, which
is sometimes called the canonical conjugate \tbf{momentum} $p$ for the
coordinate $q$.  
We can see this explicitly appear as a constant of the Euler-Lagrange equation
if the Lagrangian doesn't contain $q$, as then $\dv p t = 0$, so in this
case, $p$ is a constant of motion and our symmetry can be ``change $q$ however
we like because it doesn't matter.'' 
Regardless, from a symmetry, we have found a conserved quantity -- this, in essence, is \textbf{Noether's Theorem}! 

As a simple example, we can look at the case of a free particle not under the
influence of any potential in Cartesian coordinates (in three dimensions):
\[
	\lagr = \frac{1}{2} m v^2 = \frac{1}{2} m ( \dot{x}^2 + \dot{y}^2 + \dot{z}^2 )
\]
If we consider translating $x$ by a constant amount, then we simply get $p_x =
\pdv \lagr{\dot x} = m \dot x$ (up to a constant multiple) as a conserved
quantity... but you might recognize this as the linear momentum in the $x$
direction! Similar result holds for $y$ and $z$. 
Notice that in the absence of a potential, no external forces are acting on the
object - therefore, in the absence of an external force, we have that momentum
is conserved, 
so we get conservation of linear momentum for free! We can similarly get
a familiar conserved quantity if we apply the same idea to a free particle 
and use cylindrical/spherical coordinates, perhaps in a spherical
potential $V(r)$: 
\[
 \lagr = \frac{1}{2} m (\dot r^2 + r^2 \dot \theta^2 + r^2 \sin^2\theta \dot
 \vphi^2) - V(r)
\]
Then for $\vphi$ (being physicists, $\theta$ is the polar angle and $\vphi$ is
the azimuthal angle), we have that $p_\vphi = m r^2 \sin^2 \theta \dot \vphi = m
z^2 \dot \vphi$, which is the component of the angular momentum with respect to
the $z$-axis! We have therefore recovered two familiar conserved quantities from
mechanics from this formalization. 

\subsection{Conservation of Energy and the Hamiltonian}
How do we get to the idea of conservation of energy? The only coordinate that we
haven't really talked about homogeneity with is the time coordinate $t$, and so
by process of elimination one might assume that energy is the conserved quantity
associated with temporal invariance. However, time is special when discussing
symmetries, because we can vary the actual functions of time with our parameter
$s$ as a ``perturbation,'' but not so here. Instead, for the coordinate $q$, we
use $s$ to translate in time, via $q(s; t) = q(t + s)$. We see then that 
\[
 \dv{\lagr}s \Big|_{s = 0} = \left( \pdv \lagr q \dv q s + \pdv \lagr {\dot q}
   \dv {\dot q} s\right)\Big|_{s = 0} = \dv{} t \left( \pdv \lagr {\dot q}
 \right) q'(t) + \pdv \lagr q q''(t) = \dv \lagr t 
\]
if the Lagrangian has no explicit time dependence at all. 
Then, using the derivation above, we have that $\dv{}t (p \dot q - \lagr) = 0$, so
$\ham = p \dot q - \lagr$ appears to be the conserved quantity here. What is
this $\ham$ thing? 

If we take $\lagr = \frac{1}{2}m \dot x^2 - U(x)$ in the 1-dimensional form, $p
= m \dot x$, and so $\ham = p \dot x - \lagr = m \dot x^2 - \frac{1}{2} m \dot x^2 +
U(x) = \frac{1}{2} m \dot x^2 + U(x)$ appears to be the total energy! So it
appears that $\ham$, the \tbf{Hamiltonian}, which is the corresponding conserved
quantity for time translation, is energy. In general, this is not quite true,
especially if the Lagrangian is written in not the rest frame, but it's a
reasonable equivalence to make.  

\section{Hamiltonian Mechanics}
Since the Hamiltonian is kind of close to the Lagrangian, it might interesting
to consider maybe using the Hamiltonian (which is a slightly more familiar
quantity, and often is the total energy of the system) to get our equations of
motion instead. It turns out, we can do this, but it becomes more natural to use
the coordinates $q$ and the conjugate momentums $p$ instead of $q$ and $\dot q$
as in the Lagrangian. In this sense, $\ham$ is a function that operates over an
even-dimensional space with corresponding dimensions matching the $q$s and $p$s,
and in particular (somewhat strangely) we should take $q$ and $p$ as independent
and therefore $\dot q$ as being dependent on $q$s and $p$s. 

First, let's note $\pdv \ham p = \dot q + p \pdv {\dot q} p - \pdv{\lagr} p$ from the
chain rule. But what's $\pdv \lagr p$? 
\[
 \pdv \lagr p = \pdv \lagr q \pdv q p + \pdv \lagr {\dot q} \pdv{\dot q} p = 0 +
 p \pdv {\dot q} p \Rightarrow \pdv \ham p = \dot q
\]
Now, what about $\pdv H q$? We do the same thing, but we have to be careful
about what things are being kept constant in the partials: \[
  \pdv \ham q = \pdv{\dot q} q p + \dot q \pdv p q - \pdv \lagr q = \pdv{\dot q}
  q  p - \pdv \lagr q.
\] 
Remember that the evaluation of $\pdv \lagr q$ is holding $p$ constant, not
$\dot q$, so we have to be careful. We can do the computation with differentials
first:  
\[
  d \lagr = \pdv \lagr q \, dq + \pdv \lagr {\dot q} \, d \dot q + \pdv \lagr p
  \, dp
\] 
where each of the partials here hold the other two variables constant. 
Now set $dp = 0$, and write $d \dot q$ as $\pdv{\dot q} q \, dq$. This then
gives \[ \left(\pdv \lagr q\right)_p = \left(\pdv \lagr q\right)_{\dot q} + \left(
\pdv \lagr {\dot q} \right)_q \pdv{\dot q}q = \dot p + p \pdv{\dot q}q
\Rightarrow \pdv \ham q = - \dot p 
\] where we substitute definitions and use Euler-Lagrange. Together, the
equations $\dot p = - \pdv \ham q$ and $\dot q = \pdv \ham p$ are called
\tbf{Hamilton's equations}, and are equivalent to the Lagrangian formalism's
Euler-Lagrange equations. 

Let's do a sanity check to see if this still makes sense. Suppose $\ham$ is the
Hamiltonian for a particle with a potential $U(x)$, so $\ham = \frac{p^2}{2m} +
U(x)$. Then $-\pdv \ham x = - \dv U x = \dot p$ which is just Newton's second
law again, and $\pdv \ham p = \frac{p}{m} = \dot x$ which is just the definition
of the linear momentum. And again, if we work in say, polar coordinates in 2D,
$p_r = m \dot r$ and $p_\theta = mr^2 \dot \theta = L$, the angular momentum, so
$\ham = \frac{p_r^2}{2m} + \frac{L^2}{2mr^2} + V(r, \theta)$, and so the four
equations we get are
\[
  \frac{p_r}{m} = \dot r \quad -\pdv V r + \frac{L^2}{4mr^3} = \dot p_r \quad
  \frac{L}{mr^2} = \dot \theta \quad - \pdv V \theta = \dot L   
\] 
This seems like a weird thing to do, especially since we basically doubled the
number of equations and half of them end up appearing like definitions, so it
seems like this isn't super helpful... but with a little bit of extra structure,
it will!

\section{Poisson Brackets and Noether's Theorem (Again!)}
In general, any physical quantity we would like to measure that is not constant
can be expressed in terms of the coordinates $p$ and the momenta $q$, so an
observable $A$ is a function $A(q, p)$, where $p$ and $q$ themselves might vary
with time too. The following construct may seem unmotivated, but we'll quickly
see its utility. Define the \tbf{Poisson bracket} of two functions $\set{A, B}$ as
\[
  \set{A, B} = \pdv{A}{q} \pdv{B}{p} - \pdv{A}{p} \pdv{B}{q}
\]
The Poisson bracket has some nice properties and identities that can be proved,
but the ones we'll need are that $\set{A, A} = 0$ and $\set{A, B} = - \set{B,
A}$.

Let's do some examples -- first, we should see $\set{q, p} = 1$. We also have a
Hamiltonian $\ham$ -- what happens with the Poisson bracket of $\ham$ with the
coordinates and momenta? We can see that 
\[
  \set{q, \ham} = \pdv \ham p = \dot q \quad \set{p, \ham} = - \pdv \ham q =
  \dot p.
\] 
We can then rewrite Hamilton's equations as $\dot q = \set{q, \ham}$ and $\dot p
= \set{p, \ham}$. One might guess in general for any observable $W$, that $\dot
W$ might be amenable to the Poisson bracket too! In general, with the chain
rule, if $W$ is not explicitly dependent on time, 
\[
 \dv W t = \pdv W q \dv q t + \pdv W p \dv p t = \pdv W q \pdv \ham p - \pdv W p
 \pdv \ham q = \set{W, \ham}
\] 
so Poisson brackets with the Hamiltonian give the time derivative. If $W$
happens to be a conserved quantity, then, $\set{W, \ham} = 0$, i.e. $W$
\tbf{Poisson-commutes} with the Hamiltonian.

Where are we going with this? Let's try to build a bridge again between
conserved quantities and symmetries, but this time, we want to build a symmetry
of the Hamiltonian of our system, and the Poisson bracket will allow us to see
the duality of these perspectives even more clearly. In the language of
Hamiltonian mechanics, we consider what are called \tbf{canonical
transformations}, a $Q(\eps) = q + \eps A(q, p)$ and $P(\eps) = p + \eps B(q,
p)$, where our deviations $A$ and $B$ are functions of $q, p$, the independent
coordinates of the Hamiltonian. Suppose for our conserved quantity $W(p, q)$, we
wish for our transformation to not change $W$ up to first order. Then 
\[
 W(Q,P) = W(q + \eps A, p + \eps B) = W(q, p) + \eps A \pdv W q + \eps B \pdv W
 p 
\] 
where we ignore higher-order terms. We also want (as a part of ``canonicity'')
that our transformation should respect the existing relation $\set{Q, P} = 1$.
Ignoring higher-order terms in $\eps$ again:  
\[
  \set{Q, P} = \set{q, p} + \eps ( \set{A, p} + \set{q, B}) \Rightarrow
  \pdv A q + \pdv B p = 0
\] 
One easy way to satisfy this is when $A = \pdv V p$, $B = - \pdv V q$ for some
observable $V$, so to make the first-order deviation in $W$ to be zero, we get
that $\set{W, V} = 0$. Well, we already have $W$, so why not have $V = W$? Then
an infinitesimal transformation in coordinates that leaves $W$ invariant is $Q =
q + \eps \pdv W p$, $P = p - \eps \pdv W q$. This extends generally to larger
translations using $\dv Q \eps = \pdv W p$, $\dv P \eps = - \pdv W
q$.

Back to an infinitesimal transformation, we can see what happens to any other observable
$U$, up to first order: 
\[
 U(Q, P) = U(q, p) + \eps \left(\pdv U q \pdv W p - \pdv U p \pdv W q \right) = U(q, p) +
 \eps\set{U, W}
\] 
so $\dv U \eps = \set{U, W}$. We can use this idea to transform $U$
non-infinitesimally, holding $W$ constant, using a Taylor series. Let $P_W$ be
the operator defined by $B_W(X) = \set{X, W}$. Then: 
\[
  U(Q, P; \eps) = \sum_{k=0}^\infty \frac{\eps^k}{k!} \dnv U \eps n \Big|_{\eps = 0} =
  \sum_{k=0}^\infty \frac{\eps^k}{k!} B_W^k(X) = e^{\eps B_W}(X)
\]
so the Poisson bracket with the conserved quantity \tbf{generates} the infinite
family of transformations of $X$ holding $W$ constant. In the special case where
$W = \ham$, we see that these transformations correspond to time-translations,
since $\dv X t = \set{X, \ham}$. Moreover, when $W = p$, we have that
$\set{X, p} = \pdv X q$, so the transformation generated by $p$ gives
translations in $q$ (which goes for both $q$ that has units of length, giving
momentum, and $q$ with units of radians, giving angular momentum). In general,
we see that every conserved quantity generates a \tbf{one-parameter subgroup} of
canonical transformations acting on $(q, p)$ via the exponential map, or a
\tbf{flow} of the coordinates $(q, p)$.  

Now, we know a conserved quantity gives $\set{W, \ham} = 0$, so that $W$ is
invariant under time translations. But, we also have that $\set{\ham, W} = 0$,
so that $\ham$ is invariant under the transformations generated by $W$, which gives
a symmetry of $\ham$. This is Noether's Theorem again, but with the two notions
very clearly dual via the Poisson bracket!



\subsection{A Connection to Quantum Mechanics}
The last thing -- the Poisson bracket corresponds very closely to the commutator
of quantum mechanics fame. Consider $\set{F_1F_2, G_1G_2} = \pdv{F_1 F_2}q
\pdv{G_1 G_2}p - \pdv{F_1 F_2}p \pdv{G_1 G_2}q$. We may factor this in two ways and obtain 
\[
  \set{F_1, G_1}(F_2G_2-G_2F_2) = (F_1G_1-G_1F_1)\set{F_2, G_2}
\]
which gives 
\[
  \frac{\comm{F_1}{G_1}}{\set{F_1, G_1}} = \frac{\comm{F_2}{G_2}}{\set{F_2, G_2}}
\]
where $[F, G] = FG - GF$, assuming the $F$s and $G$s do not commute as operators. 
These have to be equal to some constant $\lambda$, since each side is
independent. This implies that $\comm{F}{G} = \lambda \set{F, G}$, which gives a
direct correspondence between the Poisson bracket of two functions and their
commutator. If these happen to be some canonical variables $q, p$, we see that 
$[q, p] = \lambda. $
If we let $\lambda = i\hbar$, this allows us to bridge the gap between classical
and quantum mechanics, as we now obtain the Heisenberg commutation relations for
position/momentum.
This small step, made first by Dirac, gives us a \tbf{canonical quantization} of classical mechanics, which is one of the first steps to rigorizing the treatment of quantum mechanics. 

\end{document} 
