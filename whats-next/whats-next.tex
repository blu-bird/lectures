\documentclass[12pt]{article}
\usepackage{blubird}
% has options [nodate, nosans, nofancy, nocolor, code]

% TITLE
\title{What's Next?}
\author{Bryan Lu}
\date{5 June 2023} % do not use if [nodate] option enabled

\begin{document}
\maketitle

\noindent {\large \textbf{Question.} What does math/doing math mean to you?}

\noindent More specific prompts:
\begin{center} 
\begin{multicols}{2}
  \begin{itemize}
    \item Why do you want to study math?
    \item Do you like math? Why? 
    \item How could you learn about math outside of school/VMT?
    \item What does being ``good'' at math mean?
    \item How diverse is the math community?  
    \item Does doing math make you ``smarter?''
    \item Why should people study math?
  \end{itemize}
\end{multicols}
\end{center}
% 5 minutes

\section{The Landscape} % (10 minutes)
\begin{center}
  \begin{minipage}{0.45\textwidth}
    Competition math categories: 
    \begin{itemize}
      \item algebra (inequalities, FEs)
      \item geometry (mostly Euclidean)
      \item combinatorics 
      \item number theory (elementary) 
    \end{itemize}
  \end{minipage}
  \begin{minipage}{0.45\textwidth}
    TJ Math (engineering requirements): 
  \begin{itemize}
    \item calculus sequence (BC + multi)  
    \item linear algebra 
    \item differential equations
    \item statistics 
  \end{itemize}
  \end{minipage}
\end{center}
This is \tbf{VERY FAR} from everything that's out there! 

\noindent Classical undergraduate categories:
\begin{center}
  \begin{minipage}{0.45\textwidth}
    Main categories of study: 
    \begin{itemize}
      \item (abstract) algebra 
      \item analysis/measure
      \item topology/geometry
    \end{itemize}
  \end{minipage}
  \begin{minipage}{0.45\textwidth}
    Less central categories:
    \begin{itemize}
      \item combinatorics
      \item logic/set theory
      \item probability/statistics
    \end{itemize}
  \end{minipage}
\end{center}
These are not clean divisions -- lots of overlap between these areas of study! 

\noindent Categorization is even trickier beyond curriculums, but some
broad fields that incorporate many of the above fields and carry
significant historical importance/development (there are many more!):
\begin{center}
  \begin{multicols}{3}
    \begin{itemize}
      \item algebraic number theory
      \item analytic number theory
      \item dynamical systems
      \item Lie theory
      \item harmonic analysis
      \item operator theory
      \item computer science
      \item foundations
      \item game theory/economics
      \item algebraic geometry
      \item algebraic topology
      \item mathematical physics 
    \end{itemize}
  \end{multicols}
\end{center}

\section{Ideas}
\tit{Myths/Misconceptions}: 
\begin{enumerate}[label=(\alph*)]
  \item Proving really \tbf{hard} theorems/solving really \tbf{hard} problems
    (misplaced emphasis) 
  \item Formalism vs. pedantry
  \item Working on your own vs. community of ideas
  \item Difficulty as a barrier to growth
  \item (there are probably tons more \dots)  
\end{enumerate}
\textbf{Core ideas} to keep in mind: 
\begin{enumerate}[label=(\arabic*)]
  \item Connecting seemingly unrelated objects/procedures.  
  \item Extending/generalizing ideas we already know. 
  \item Analyzing when/why patterns break and seeking complexity.
  \item Staying grounded with ``standard'' examples/calculations. 
\end{enumerate}
\tbf{Goal:} I want to show you these ideas in action with examples and stories!  
\section{Examples/Stories}
The rest of this lecture will proceed in a ``choose your own adventure'' format:
For each of the first three \tbf{Core Ideas}, you will get to pick one of the
two stories to talk about via democracy. Some topics are, by my estimation, a
little more difficult/inaccessible, so these have been marked with a $\bigstar$. 

\subsection{Connections}
Both of the following stories/examples revolve around the use of graphs combined
with algebra in the pursuit of various avenues of study:
\begin{itemize}
  \item \textit{How to Tell Spaces Apart (Simplicial Homology)} --
    We will use graphs as a model for describing various topological spaces
    (spheres, toruses, Klein bottles) and use an extension of the Euler
    characteristic with linear algebra to distinguish spaces from each other. 
  \item ($\bigstar$) \tit{Word Problems (Geometric Group Theory)} -- We
    will take an alternate view of groups and study graphs induced by a group's
    structure, and use these graphs to explain why deciding whether an element
    in a group is the identity element is unsolvable. 
\end{itemize}


%\begin{itemize}
%  \item Galois theory and correspondences (examples in different contexts
%    too; covering spaces too?) 
%\subsubsection{Yet Another Galois Correspondence}
% this is too hard
%  \item Euler characteristic connecting to algebraic topology, simplicial homology
%
%  \item Lie algebra - group correspondence
% this is too hard 
%  \item solving the word problem 
%\end{itemize}

\noindent \tbf{(Enter break here, informal time for questions)}

\subsection{Generalizations}
The following two stories describe generalizations of ideas and
concepts you may have seen before via the formalisms of analysis and geometry:  
\begin{itemize}
  \item $(\bigstar)$ \tit{Doing Calculus Anywhere (Manifolds,
    Differential Forms)} -- One of the staples of calculus is the Fundamental
    Theorem of Calculus, which has many further generalizations further afield
    in multivariable calculus. Let's see why these are actually all the same
    thing! 

  \item \tit{A Case Study in Volume (Measure Theory)} -- So you
    think you know what volume is? Here we test your intuitive understanding of
    volume and use edge cases as a case study for generalizing what you might
    think of as ``volume.'' 
\end{itemize}

\subsection{Breaking and Repairing}
These final two stories describe ways in which natural constructions might fail
to behave as expected, and when possible, how we can make sense of the complexity
that arises as a result. 

\begin{itemize}
  \item \tit{I Forgot How to Factor (Algebraic
    Number Theory)} -- We will try solving some Diophantine equations in some
    creative ways, and realize that I don't really know how to prime-factorize
    things anymore. Can we fix that? 

  \item \tit{The Answer Is Not True (Analysis,
    Foundations)} -- Because everyone loves the true/false section of a math 
    exam, we're going to have a true/false party!

    Wait, I forgot to take the spoiler out of the section header... fine, to
    make it more interesting, be prepared to explain your answers!
\end{itemize}

%\begin{itemize}
%  \item Unique factorization, algebraic number theory  
%  \item Bad/pathological examples from topology/analysis
%\subsubsection{(another game of true/false)}
%  \item Axiom of choice stuff? What is a feature and what is a bug?  
%\end{itemize}

\section*{Questions?}


\end{document} 
